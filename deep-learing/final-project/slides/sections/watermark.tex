\section{Watermarking Overview}
\subsection{A History of Watermarks}
\begin{frame}{A History of Watermarks}
    \begin{itemize}
        \item Watermarking is the practice of altering a work to embed a message about that work.
        \item Paper watermarks first appeared in Italy in 1282
        \item Used with precise purposes by 18th century: trademarks and manufacturing records.
    \end{itemize}
\end{frame}


\subsection{Watermarking Models}
\begin{frame}{Watermarking Models}
    Watermark types
    \begin{itemize}
        \item Visible: insert a logotype or a text into the image
        \item Invisible: embed an imperceptible signal into the image
    \end{itemize}
\end{frame}

\begin{frame}{Visible Watermarking}
    Requirements
    \begin{itemize}
        \item Should be perceptible in both gray and colored image
        \item Should be perceptible in pixels with different characteristics: texture, plain and edge
        \item Should not be too obtrusive
        \item Should be robust against several common attacks
        \item Watermark embedding process should be automatic for kinds of images.
    \end{itemize}
\end{frame}

\begin{frame}{Visible Watermarking - An Example}
    The watermarked image is obtained pixel-by-pixel by
    $$p'(i,j)=p(i,j)+\alpha
    (i,j)w(i,j),$$
    where
    \begin{itemize}
        \item $p(i,j)$ is the original pixel
        \item $\alpha(i,j)$ is the watermark strength
        \item $w(i,j)\in\{0,1\}$ is the watermark value
    \end{itemize}
\end{frame}

\begin{frame}{Visible Watermarking - An Example}
    \begin{itemize}
        \item The gray scale resolution of human
eyes is decreased in higher and lower gray conditions.
        \item Watermark strength should match.

        \begin{figure}
            \centering
            \includegraphics[width=0.5\linewidth]{image.png}
            \caption{The gray scale resolution of human eyes}
            \label{fig:enter-label}
        \end{figure}
    \end{itemize}
\end{frame}

\begin{frame}{Visible Watermarking - An Example}
    \begin{itemize}
        \item Approximate the watermark strength by lines \cite{yu2013new}
        $$\alpha = \left\{\begin{array}{l}      -\dfrac{1}{8}p+6,\,\,p\in[0,32]\\
   -\dfrac{1}{32}p+3,\,\,p\in[33,64]\\
   \dfrac{1}{96}p+\dfrac{1}{3},\,\,p\in[65,255]\\\end{array}\right.$$

    \end{itemize}
\end{frame}


\begin{frame}{Visible Watermarking - An Example}
    \begin{figure}
        \centering
        \includegraphics[width=0.7\linewidth]{img/visible-watermark.png}
        \caption{Watermarked images by \cite{yu2013new}}
        \label{fig:enter-label}
    \end{figure}
\end{frame}


% \begin{frame}{Invisible Watermarking}
% Broad groups: communication-based and geometric

%     \begin{figure}
%     \centering
%     \includegraphics[width=0.9\linewidth]{img/communication-based.png}
%     \caption{Watermarking as communications with side information at the transmitter}
%     \label{fig:enter-label}
% \end{figure}
% \end{frame}

% \begin{frame}{Invisible Watermarking}
% Broad groups: communication-based and geometric
%     \begin{figure}
%         \centering
%         \includegraphics[width=0.9\linewidth]{img/extractor.png}
%         \caption{Geometric watermarking: General two-step outline of a watermark detector.}
%         \label{fig:enter-label}
%     \end{figure} 
% \end{frame}

% \begin{frame}{Invisible Watermarking}
% Broad groups: communication-based and geometric
%    \begin{figure}
%        \centering
%        \includegraphics[width=0.9\linewidth]{img/embedder.png}
%        \caption{Geometric watermarking: General three-step outline of a watermark embedder.
% }
%        \label{fig:enter-label}
%    \end{figure}
% \end{frame}

% \subsection{Visible Watermark Removal}
% \begin{frame}{Visible Watermark Removal}
% Large-Scale Visible Watermark Detection and Removal with Deep
% Convolutional Networks \cite{cheng2018large} (PRCV - 2018)
%    \begin{figure}
%        \centering
%        \includegraphics[width=0.9\linewidth]{img/r1pipeline.png}
%        \caption{The proposed pipeline}
%        \label{fig:enter-label}
%    \end{figure}
% \end{frame}

% \begin{frame}{Visible Watermark Removal}
%    \begin{figure}
%        \centering
%        \includegraphics[width=0.9\linewidth]{img/r1architechture.png}
%       \caption{The proposed architecture}
%        \label{fig:enter-label}
%    \end{figure}
% \end{frame}

% \begin{frame}{Visible Watermark Removal}
%    \begin{figure}
%        \centering
%        \includegraphics[width=0.9\linewidth]{img/r1result.png}
%        \caption{Result of watermark removal}
%        \label{fig:enter-label}
%    \end{figure}
% \end{frame}

% \begin{frame}{Visible Watermark Removal}
% Generative adversarial networks model for visible watermark removal \cite{cao2019generative} (IET Image Processing - 2019)
%    \begin{figure}
%        \centering
%        \includegraphics[width=0.9\linewidth]{img/r2architechture.png}
%        \caption{The proposed architecture}
%        \label{fig:enter-label}
%    \end{figure}
% \end{frame}

% \begin{frame}{Visible Watermark Removal}
%    \begin{figure}
%        \centering
%        \includegraphics[width=0.9\linewidth]{img/r2result.png}
%        \caption{Result of watermark removal}
%        \label{fig:enter-label}
%    \end{figure}
% \end{frame}

% \begin{frame}{Visible Watermark Removal}
% Blind Visual Motif Removal from a Single Image \cite{hertz2019blind} (CVPR - 2019)
%    \begin{figure}
%        \centering
%        \includegraphics[width=0.9\linewidth]{img/r3pipeline.png}
%        \caption{The proposed pipeline}
%        \label{fig:enter-label}
%    \end{figure}
% \end{frame}

% \begin{frame}{Visible Watermark Removal}
%    \begin{figure}
%        \centering
%        \includegraphics[width=0.75\linewidth]{img/r3result.png}
%        \caption{Output visualization. Top: test images embedded with unseen motifs. Middle: our reconstructed background images. Bottom: our reconstructed visual motifs. More results are presented in the supplementary material}
%        \label{fig:enter-label}
%    \end{figure}
% \end{frame}

% \begin{frame}{Visible Watermark Removal}
%    \begin{figure}
%        \centering
%        \includegraphics[width=0.9\linewidth]{img/r3result2.png}
%        \caption{Result of watermark removal}
%        \label{fig:enter-label}
%    \end{figure}
% \end{frame}