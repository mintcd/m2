\section{Introduction}

\begin{frame}{Motivation}
    \begin{itemize}
        \item A high-performance computing (HPC) system consists of nodes. A number of nodes are allocated when a user submits a tasks and recollected after the task is executed.
        \item If there is no node available, the next task comes the the waiting queue.
        \item Task scheduling is vital in high-performance computing systems in order to reduce waiting time (the time from submission to starting execution), execution time or total time. Furthermore is there energy.
    \end{itemize}
\end{frame}

\begin{frame}{Problem Statement}
    \begin{itemize}
        \item Train schedulers to decide the number of allocated nodes using different techniques: RNN and Deep RL with objective to be the waiting time.
        \item Compare with some common policies.
    \end{itemize}
\end{frame}